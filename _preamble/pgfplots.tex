\usepackage{pgfplots,tikz-3dplot}
\usetikzlibrary{
	3d,
	arrows,
	arrows.meta,
	calc,
	decorations,
	decorations.markings,
	external,
	intersections,
	math,
	matrix,
	patterns,
	plotmarks,
	spy,
	through,
}
\usepgfplotslibrary{
	colorbrewer,
	patchplots,
	fillbetween
}


%% Externalization
\tikzexternalize[prefix=_tikz/]  % Externalize every chart
% \tikzset{external/only named=true}
% \newcommand{\inputtikzext}[2]{\tikzsetnextfilename{#1}\input{#2}}

%% Custom styles
\pgfplotsset{
	compat=newest,
	axis line style={>=latex},
	tick label style={font=\scriptsize},
	label style={font=\footnotesize},
	x tick scale label style={
			at={(rel axis cs:1,0,0)},
			anchor=south west,
			inner sep=1pt},
	y tick scale label style={
			at={(rel axis cs:0,1,0)},
			anchor=south west,
			inner sep=1pt},
	z tick scale label style={
			at={(rel axis cs:0,0,1)},
			anchor=south west,
			inner sep=1pt},
	legend style = {
			cells={anchor=west},
			font=\footnotesize,
			draw=none,
			fill=none,
		},
}

\pgfkeys{/pgf/number format/1000 sep={}}

% Block schemes
\tikzset{
	>=latex,
	every node/.style={font=\footnotesize},
	block/.style = {
			rectangle,
			rounded corners,
			draw=black,
			fill=white,
			node distance=3.5cm,
			minimum width=2cm,
			minimum height=1cm,
			text centered,
			font=\footnotesize,
			inner sep=5pt,
		},
	decision/.style = {
			diamond,
			minimum width=2cm,
			minimum height=1cm,
			text centered,
			draw=black,
			fill=white,
			font=\footnotesize,
		},
	inputoutput/.style = {
			trapezium,
			trapezium left angle=70,
			trapezium right angle=110,
			minimum width=2cm,
			minimum height=1cm,
			text centered,
			draw=black,
			fill=white,
			font=\footnotesize,
		},
	dot/.style ={
			circle,
			draw=black,
			fill=black,
			inner sep=1pt,
		},
}


%% Center of mass symbol of given radius
% Example: \node at (0,0,0) {\centerofmass{0.4em}};
\newcommand*{\centerofmass}[1]{%
	\tikz[radius=#1] {%
		\fill (0,0) -- ++(#1,0) arc [start angle=0,end angle=90] -- 
		++(0,-2*#1) arc [start angle=270, end angle=180];%
		\draw (0,0) circle;%
	}%
}

% Draw arc from center
% Syntax: [draw options] (center) (initial angle:final angle:radius)
% Example: \centerarc[red,thick](0,0)(5:85:1)(nodeoptions)(nodetext);
\def\centerarc[#1](#2)(#3:#4:#5)(#6)(#7)%
{ \draw[#1] ($(#2)+({#5*cos(#3)},{#5*sin(#3)})$) arc (#3:#4:#5) node[#6] {#7};}