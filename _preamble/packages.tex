% File Management
%% ---------------------------------------------------

\usepackage{standalone, pdfpages}

%% Page layout, language and font
%% ---------------------------------------------------

\usepackage[utf8]{inputenc}
\usepackage[english,italian]{babel}
\usepackage[margin=1cm, vmargin=2cm]{geometry}
\setlength\columnsep{18pt}
\setlength{\parindent}{0pt} % Remove paragraph indentation

\usepackage{newtxtext}

\usepackage{
	enumitem,
	booktabs,
	csquotes,
	ragged2e}

%% Colors and Lexical
%% ---------------------------------------------------

\usepackage{graphicx, xcolor, xspace, rotating}

%% MATLAB color scheme
%% ---------------------------------------------------------------
\definecolor{myblue}{rgb}{0.00000,0.44700,0.74100}%
\definecolor{myred}{rgb}{0.85000,0.32500,0.09800}%
\definecolor{myyellow}{rgb}{0.92900,0.69400,0.12500}%
\definecolor{mypurple}{rgb}{0.49400,0.18400,0.55600}%
\definecolor{mygreen}{rgb}{0.46600,0.67400,0.18800}%
\definecolor{mylightblue}{rgb}{0.30100,0.74500,0.93300}%
\definecolor{mydarkred}{rgb}{0.63500,0.07800,0.18400}%

%% TU Delft color scheme
%% ---------------------------------------------------------------
\definecolor{tudelft-cyan}{cmyk}{1,0,0,0}
\definecolor{tudelft-black}{cmyk}{0,0,0,1}
\definecolor{tudelft-white}{cmyk}{0,0,0,0}
\definecolor{tudelft-sea-green}{cmyk}{0.54,0,0.32,0}
\definecolor{tudelft-green}{cmyk}{1,0.15,0.4,0}
\definecolor{tudelft-dark-blue}{cmyk}{1,0.66,0,0.4}
\definecolor{tudelft-purple}{cmyk}{0.98,1,0,0.35}
\definecolor{tudelft-turquoise}{cmyk}{0.82,0,0.21,0.08}
\definecolor{tudelft-sky-blue}{cmyk}{0.45,0,0.06,0.06}
\definecolor{tudelft-lavendel}{cmyk}{0.45,0.2,0,0.07}
\definecolor{tudelft-orange}{cmyk}{0.02,0.56,0.84,0}
\definecolor{tudelft-warm-purple}{cmyk}{0.58,1,0,0.02}
\definecolor{tudelft-fuchsia}{cmyk}{0.19,1,0,0.19}
\definecolor{tudelft-bright-green}{cmyk}{0.36,0,1,0}
\definecolor{tudelft-yellow}{cmyk}{0.02,0,0.54,0}

%% Abbreviations
%% ---------------------------------------------------------------
\newcommand*{\eg}{e.g.\@\xspace}
\newcommand*{\ie}{i.e.\@\xspace}
\newcommand*{\etc}{etc.\@\xspace}
\newcommand*{\wrt}{w.r.t.\@\xspace}
\newcommand{\quotes}[1]{``#1''}

%% Bold symbols
%% ---------------------------------------------------------------
\newcommand{\hmmax}{0} % Reduces heavy symbols
\newcommand{\bmmax}{0} % Reduces bold symbols

%% Mathematics and math symbols
%% ---------------------------------------------------------------

\usepackage{
	amsmath,
	amssymb,
	mathtools,
	bm,
	xfrac,
	calc,
	smartdiagram,
	newtxmath}

\usepackage[TX]{ar}

\newcommand{\twovec}[2]{%
	\left(\!\!
	\begin{array}{c}
		#1 \\ #2
	\end{array}
	\!\!\right)}
\newcommand{\threevec}[3]{%
	\left(\!\!
	\begin{array}{c}
		#1 \\ #2 \\ #3
	\end{array}
	\!\!\right)}

%% Operators
%% ---------------------------------------------------------------
\newcommand{\pdiff}[2]{\frac{\partial #1}{\partial #2}}
\newcommand{\dd}{\ensuremath{\,\mathrm{d}}}
\newcommand{\ddt}{\ensuremath{\,\dd t}}
\newcommand{\ddx}{\ensuremath{\,\dd x}}
\newcommand{\odiff}[2]{\frac{\dd #1}{\dd #2}}
\newcommand{\eps}{\varepsilon}
\newcommand{\sign}{\mathrm{sign}}
\newcommand{\erf}{\mathrm{erf}}
\newcommand{\overbar}[1]{\mkern 1.5mu\overline{\mkern-1.5mu#1}\mkern 1.5mu}
\newcommand{\subjto}{\ensuremath{\text{subj. to}}}

%%  Delimiters
%% ---------------------------------------------------------------
\DeclarePairedDelimiter{\abs}{\lvert}{\rvert}
\DeclarePairedDelimiter{\norm}{\lVert}{\rVert}
\DeclarePairedDelimiter{\rowvec}{\lbrack}{\rbrack}
\DeclarePairedDelimiter{\rowset}{\lbrace}{\rbrace}

% Swap the definition of \delimiter*, so that \delimiter{}
% resizes the size of the brackets, and the starred version does not.
\makeatletter
%
\let\oldabs\abs
\def\abs{\@ifstar{\oldabs}{\oldabs*}}
%
\let\oldnorm\norm
\def\norm{\@ifstar{\oldnorm}{\oldnorm*}}
%
\let\oldrowvec\rowvec
\def\rowvec{\@ifstar{\oldrowvec}{\oldrowvec*}}
%
\let\oldrowset\rowset
\def\rowset{\@ifstar{\oldrowset}{\oldrowset*}}
%
\makeatother

%% Units
%% ---------------------------------------------------------------
\usepackage{siunitx}
\sisetup{
	per-mode=symbol,
	inter-unit-product = \ensuremath{\! \cdot{} \!},
}

\DeclareSIUnit[]{\degree}{deg}
\DeclareSIUnit[]{\inch}{in}
\DeclareSIUnit[]{\foot}{ft}
\DeclareSIUnit[]{\mile}{mi}
\DeclareSIUnit[]{\nmile}{nmi}
\DeclareSIUnit[]{\rpm}{rpm}
\DeclareSIUnit[]{\mph}{mph}
\DeclareSIUnit[]{\knot}{kts}
\DeclareSIUnit[]{\pound}{lb}
\DeclareSIUnit[]{\slug}{slug}
\DeclareSIUnit[]{\ton}{ton}
\DeclareSIUnit[]{\pint}{pt}
\DeclareSIUnit[]{\quart}{qt}
\DeclareSIUnit[]{\igal}{gal}
\DeclareSIUnit[]{\usgal}{US \ gal}
\DeclareSIUnit[]{\hour}{hr}
\DeclareSIUnit[]{\minute}{min}
\DeclareSIUnit[]{\kiloforce}{kgf}
\DeclareSIUnit[]{\poundforce}{lbf}
\DeclareSIUnit[]{\poundfoot}{\poundforce\foot}
\DeclareSIUnit[]{\psi}{psi}
\DeclareSIUnit[]{\atm}{atm}
\DeclareSIUnit[]{\inchMercury}{\si{\inch} \ Hg}
\DeclareSIUnit[]{\britishthermalunit}{Btu}
\DeclareSIUnit[]{\ihorsepower}{hp}
\DeclareSIUnit[]{\mhorsepower}{cv}
\DeclareSIUnit[]{\fahrenheit}{\text{\textdegree}\mathrm{F}}
\DeclareSIUnit[]{\rankine}{\text{\textdegree}\mathrm{R}}

\DeclareSIUnit[]{\passenger}{\mathrm{pax}}

%% Figures and tables
%% ---------------------------------------------------------------

\usepackage{
	multicol,
	multirow,
	makecell,
	colortbl}

\usepackage{caption, subcaption}
\newlength{\myfigh}
\newlength{\myfigw}
\setlength{\myfigh}{1\linewidth}
\setlength{\myfigw}{1\linewidth}

\captionsetup[table]{
	position=top,
	font=footnotesize,
	labelfont={bf,rm,footnotesize},
	labelsep=space,
	singlelinecheck=true,
	format=hang
}

\captionsetup[figure]{
	position=bottom,
	font=footnotesize,
	labelfont={bf,rm,footnotesize},
	labelsep=space,
	singlelinecheck=true,
	format=hang
}

\captionsetup[subfigure]{
	position=bottom,
	font=footnotesize,
	labelformat=brace,
	labelfont={bf,rm,footnotesize},
	labelsep=space,
	singlelinecheck=true,
	format=hang,
}

\newcolumntype{L}{>{$}l<{$}} % left-aligned, automatic math mode
	\newcolumntype{C}{>{$}c<{$}} % center-aligned, automatic math mode
\newcolumntype{R}{>{$}r<{$}} % center-aligned, automatic math mode

% Trick Supertabular into Multicols in new command
% e.g.
%	\begin{multicols*}{2}
%		\TrickSupertabularIntoMulticols
%		\begin{supertabular}{|l|l|}
%			\tablebody
%		\end{supertabular}
%	\end{multicols*}
\makeatletter
\let\mcnewpage=\newpage
\newcommand{\TrickSupertabularIntoMulticols}{%
	\renewcommand\newpage{%
		\if@firstcolumn
			\hrule width\linewidth height0pt
			\columnbreak
		\else
			\mcnewpage
		\fi
	}%
}
\makeatother

%% Cross references
%% ---------------------------------------------------------------

\usepackage[perpage]{footmisc}
\usepackage[hidelinks]{hyperref}

\usepackage[noabbrev, nameinlink]{cleveref}
\creflabelformat{equation}{#2#1#3}
% \crefname{equation}{Eq.}{Eqs.}
% \crefname{figure}{Fig.}{Figs.}
% \crefname{tabular}{Tab.}{Tabs.}