% Reference:
%
% \newglossaryentry{main:entry}{
%   type=main,
%   name={},						% appears as the entry title in the Glossary recap
%   description={},			% appears as the entry description in the Glossary recap
%   text={},						% appears when using \gls{main:entry} in the text
% }

%A
\newglossaryentry{main:as}{
	type=main,
	name={Action Space},
	description={%
		It is a Cartesian axis system in the $\mathbb{R}^{\gsubbold{N}{A}}$ space, with an aerodynamic control action --- force or moment, dimensional or dimensionless --- varying on each axis. 
		Each resultant control action generated by a combination of \gls{main:ce} displacements is represented by a point in Action Space.
		The \gls{main:eas} and the \gls{main:aas} belong to Action Space.
		It is first introduced in \Cref{sec:ca:geometric-interpretation}.
		},
	text={Action Space},
}


\newglossaryentry{main:acs}{
	type=main,
	name={Admissible Controls Set},
	description={%
			It is the set of all possible effectors displacements admitted by the physical realization of the \gls{main:fcs} and the aircraft architecture.
			It is a subset of \gls{main:cs}.
			If the effectors positions are simply bounded, the \gls{main:acs} is a hyper-rectangle in \gls{main:cs}.
			It maps to the \gls{main:eas} through a linear application based on the effectiveness matrix \gls{B}.
			It is first introduced in \Cref{sec:ca:geometric-interpretation}.
		},
	text={Admissible Controls Set},
}

\newglossaryentry{main:aas}{
	type=main,
	name={Attainable Actions Set},
	description={%
			It is the subset of the \gls{main:eas} which a \gls{main:ca} method can trace to feasible positions of the effectors.
			Control objectives of the \gls{main:eas} which are outside of the \gls{main:aas} are unattainable by the given \gls{main:ca} algorithm, despite being actually attainable by the control power available to the aircraft.
			This is only dependent on the formulation of the \gls{main:ca} problem.
			In classic \gls{main:ca} literature, it is usually referred to as the \quotes{Attainable Moment Set}.
			It is first introduced in \Cref{sec:ca:attainable-moment-set}, and its geometry is exploited in \Cref{ch:design,ch:trim}.
		},
	text={Attainable Actions Set},
}


%B



%C
\newglossaryentry{main:ca}{
	type=main,
	name={Control Allocation},
	description={%
			It is a technique used to calculate the optimal effectors displacements necessary to achieve an assigned control goal.
			The control goal is usually prescribed by an overarching control law or directly by the pilot.
			The objective of the \gls{main:ca} problem is to associate a given point in \gls{main:as}, either inside or outside of the \gls{main:eas}, to a point belonging to the \gls{main:acs}.
			\gls{main:ca} methods can be in the form of analytical functions or iterative algorithms, and almost always move from the formulation of an optimization problem.
			Solutions can be obtained in closed form expression or by convergence of numerical methods.
			The most relevant aspects of the \gls{main:ca} problem are presented in \Cref{ch:control-allocation}.
		},
	text={Control Allocation},
}

\newglossaryentry{main:ce}{
	type=main,
	name={Control Effectors},
	description={%
	Control effectors are intended as different types of physical devices which are able to generate control forces and/or moments on the aircraft, in a sufficiently small time span, for the sake of maneuvering flight.
	Control effectors include aerodynamic control surfaces, throttle and thrust vectoring systems; but also devices for active flow control, variable camber airfoils, or integrated architectures such as the propulsive empennage concept employed by the \gls{duuc} aircraft.
	It is a broad definition, exploited whenever possible throughout the entire dissertation for the sake of generality.
	It has been first introduced in \Cref{ch:introduction}, and heavily used in \Cref{ch:control-allocation}.
		},
	text={control effector},
	plural={control effectors},
}

\newglossaryentry{main:ccp}{
	type=main,
	name={Control Center of Pressure},
	description={%
			It is the application point of lift generated exclusively by \glspl{main:ce}.
			It is also related to the aircraft \glsentrylong{icr}.
			Its longitudinal position heavily influences the characteristics of the longitudinal response of the aircraft, and determines the possibility to achieve \gls{main:dlc} more or less effectively.
			Its role in aircraft control is described in \Cref{sec:intro:role-of-the-ccop}.
			Its defition is then exploited in \Cref{ch:transient} to formulate a novel \gls{main:ca} method.
		},
	text={Control Center of Pressure},
}


\newglossaryentry{main:cs}{
	type=main,
	name={Control Space},
	description={%
		It is a Cartesian axis system in the $\mathbb{R}^{\gsub{N}{ubold}}$ space, with a \gls{main:ce} displacement varying on each axis. 
		Each combination of \glspl{main:ce} displacements is represented by a point in Control Space.
		The \gls{main:acs} belongs to Control Space.
		It is first introduced in \Cref{sec:ca:geometric-interpretation}.
		},
	text={Control Space},
}

\newglossaryentry{main:cwpi}{
	type=main,
	name={Constrained Weighted Pseudo Inverse},
	description={%
			It is an alternative formulation of the \gls{main:wpi} \gls{main:ca} method, including upper and lower bounds for the effectors directly in the optimization problem.
			This addition makes the problem formuation inherently non-linear.
			For this reason, differently from the \gls{main:wpi} case, the analytic solution is not available, and the \gls{main:ca} problem must be solved via an iterative algorithm.
			It is presented in \Cref{sec:ca:wpi} and used in \Cref{ch:design}.
		},
	text={Constrained Weighted Pseudo Inverse},
}


%D
\newglossaryentry{main:da}{
	type=main,
	name={Direct Allocation},
	description={%
			It is a \gls{main:ca} method formulated on the basis of the geometry of the \gls{main:eas}.
			By definition, \gls{main:da} is hence capable to attain all of the prescribed control objectives within the \gls{main:eas} or on its boundary.
			It takes into account effectors saturation limits and preserves directionality in \gls{main:as} for unattainable desired moments, but does not allow any prioritization of effectors.
			It must be solved by an iterative algorithm, and the most efficient one uses linear programming techniques.
			It is introduced in \Cref{sec:ca:direct-allocation}, implemented in \Cref{ch:design,ch:trim} and considered in \Cref{ch:transient}.
		},
	text={Direct Allocation},
}

\newglossaryentry{main:dlc}{
	type=main,
	name={Direct Lift Control},
	description={%
			It is defined as the capability to use \glspl{main:ce} to alter the aircraft lift \quotes{without, or largely without, significant change in the aircraft incidence, and ideally is meant not to generate pitching moment}.
			In the case of \gls{main:dlc}, the lift unbalance caused by \glspl{main:ce} can rapidly result in a translational acceleration, and hence in a variation of the aircraft trajectory, through alteration of the flight path angle.
			It features more rapid trajectory response, and no non-minimum phase behavior, as compared to \gls{main:cpc}.
			It is introduced in \Cref{sec:intro:dlc}.
		},
	text={Direct Lift Control},
}


% E 
\newglossaryentry{main:eas}{
	type=main,
	name={Effective Actions Set},
	description={%
			It is the set of all control forces and moments that can be, in principle, generated by the effectors.
			In general, it is a function of the \gls{main:acs} and of a given flight condition.
			For a linear aerodynamic model, if the control effectiveness matrix \gls{B} is constant and the \gls{main:acs} is a convex set, the \gls{main:eas} is a bounded convex polytope in \gls{main:as}.
			In classic \gls{main:ca} literature, it is usually referred to as \quotes{Largest Attainable Moment Set}.
			It is introduced in \Cref{sec:ca:geometric-interpretation} and further discussed in \Cref{sec:ca:attainable-moment-set}.
		},
	text={Effective Actions Set},
}


%F
\newglossaryentry{main:fcs}{
	type=main,
	name={Flight Control System},
	description={%
		It is the set of aircraft components, mechanisms and devices that serve to control the aircraft attitude and direction in flight.
		It links the pilot inputs to the \glspl{main:ce}, or commands the \glspl{main:ce} according to prescribed feedback signals and control laws.
		The most basic \gls{main:fcs} architecture consists of mechanical linkages which gear and gang \glspl{main:ce}, and directly connect them to the pilot input in the cockpit.
		More advanced \gls{main:fcs} architectures command the effectors through electronic interfaces (fly-by-wire) regulated by flight computers, and/or make use of different types of actuators to overcome the hinge moments due to great dynamic pressure.
		Some general considerations on the architecture of the \gls{main:fcs}, and specifically on ganging control surfaces, are made in \Cref{sec:intro:opportunity-with-redundant-effectors,sec:ca:redundant-effectors,sec:ca:gen-inverse}. 
		The baseline \gls{main:fcs} architecture used for all flight simulations used in the present dissertation, if not specified otherwise, is presented in \Cref{sec:models:flight-control-system}.
		A particular \gls{main:fcs} architecture has been developed in \Cref{sec:transient:fcs} for comparing the transient response of the \gls{main:prp} for different positions of the \gls{main:ccp}.
		},
	text={Flight Control System},
}

\newglossaryentry{main:fqs}{
	type=main,
	name={Flying Qualities},
	description={%
			They indicate the characteristics of an aircraft concerning aspects such as equilibrium, static stability, dynamic stability, control and dynamic response.
			All of these disciplines have to be studied to determine if the aircraft can be flown appropriately and safely in steady and maneuvering flight,
			\quotes{regardless of design implementation or \gls{main:fcs} mechanization}.
			\Gls{main:fqs} can be measured objectively through flight simulation and/or flight testing.
			Their values should comply with specific, quantitative criteria in order to qualify the aircraft performance as satisfying, acceptable or unacceptable.
			In these regards, they differ from handling qualities, which involve the pilot maneuvering experience and have to be evaluated with more subjective methods.
			Reference \Gls{main:fqs} requirements are used in \Cref{ch:design}, and specifically presented in \Cref{sec:design:hfq}.
		},
	text={Flying Qualities},
}

%G



%H



%I



%J



%K



%L



%M



% N


%O



%P
\newglossaryentry{main:psinv}{
	type=main,
	name={Pseudo Inverse},
	description={%
			It is a particular formulation of the \gls{main:wpi} \gls{main:ca} method, which does not prioritize the \glspl{main:ce}, by using an identity weighting matrix \gls{W}, and uses their neutral position as the reference one.
			In the same way as for the \gls{main:wpi}, this formulation also results in a closed form solution, but its \gls{main:aas} is always smaller than the corresponding \gls{main:eas}.
			It is first presented in \Cref{sec:ca:wpi} and used in \Cref{ch:design,ch:transient}.
		},
	text={Pseudo Inverse},
}

\newglossaryentry{main:prp}{
	type=main,
	name={PrandtlPlane},
	description={%
	It is an innovative aircraft configuration, featuring a staggered box-wing geometry, and designed to be operated for commercial transport in the short and medium range segment. 
	Thanks to the optimum induced drag properties of the box-wing, it has been identified as one of the possible solutions towards more sustainable aviation in the near future.
	The double wing architecture allows to position multiple control surfaces, which are redundant for the longitudinal and lateral control of the aircraft.
	As they fall both in front and behind the aircraft \gls{cg}, the \gls{main:prp} represents a good test subject to explore innovative applications of \gls{main:ca} methods and \gls{main:dlc}.
	It is first introduced in \Cref{ch:introduction} and presented in more detail in \Cref{sec:models:prp}.
		},
	text={PrandtlPlane},
}



%Q



%R
\newglossaryentry{main:rms}{
	type=main,
	name={Root Mean Square},
	description={%
	It is defined as the square root of the arithmetic mean of the squares of a given set of values, as reported in the equation below.
	\begin{equation*}
		\gsubrm{x}{rms} = \sqrt{\frac{1}{\gls{N}}
		\left( x^2_1 + x^2_2 + \ldots + x^2_{\gls{N}}\right)}
	\end{equation*}
	The \gls{main:rms} error, where $x$ is the error, is a common measure of the difference between two data sets, one of which can generally be treated as reference.
	The latter is used to evaluate the aircraft tracking performance in \Cref{ch:transient}.
		},
	text={Root Mean Square},
}



%S



%T
\newglossaryentry{main:tpc}{
	type=main,
	name={Tail Pitch Control},
	description={%
			It refers to the use \glspl{main:ce} with the aim of generating moments about the aircraft \gls{cg}.
			For all airplanes with conventional architecture, a tail elevator is used to generate a small, dislocated control lift, which is relevant only insofar it produces a significant pitch moment and gives raise to some angle of attack dynamics.
			Tail Pitch Control is a very indirect control technique, which results in a time delay of the trajectory response, and non-minimum phase behavior in the vertical axis dynamics.
		},
	text={Conventional Pitch Control},
}


%U



%V



%W


\newglossaryentry{main:wpi}{
	type=main,
	name={Weighted Pseudo Inverse},
	description={%
		It is a \gls{main:ca} method based on a generalized inverse formulation. 
		It results from an optimization problem to minimize control effort while ensuring that the \gls{main:ca} problem is satisfied. 
		It uses a weighting matrix \gls{W} to prioritize the \glspl{main:ce}, and a reference position of the effectors \gsub{ubold}{ref} to drive the \gls{main:ca} problem to a preferred solution.
		Both of these parameters need to be prescribed according to some criterion, and can be shaped according to the application of interest.
		This \gls{main:ca} formulation results in a closed form solution, but its \gls{main:aas} is always smaller than the corresponding \gls{main:eas}, as for all generalized inverses.
		It is first presented in \Cref{sec:ca:wpi}, and an original implementation of it is proposed in \Cref{ch:transient}.
		},
	text={Weighted Pseudo Inverse},
}


%X



%Y



%Z





